%%%%%%%%%%%%%%%%%%%%%%%%%%%%%%%%%%%%%%%%%%%
%%%%%%%%%%%%%%%%%%%%%%%%%%%%%%%%%%%%%%%%%%%
%%%%%%%%%%%%HYPHENATION%%%%%%%%%%%%%%%%%%%%
%%%%%%%%%%%%%%%%%%%%%%%%%%%%%%%%%%%%%%%%%%%
%%%%%%%%%%%%%%%%%%%%%%%%%%%%%%%%%%%%%%%%%%%

\providehyphenmins{russian}{{3}{3}}

\hbadness=10000
\interfootnotelinepenalty=10000
\doublehyphendemerits=9999
\pretolerance=100
\tolerance=1500
\emergencystretch=\maxdimen
\hyphenpenalty=6000
\brokenpenalty9000\relax
\clubpenalty=9996
\widowpenalty=9999

\setcounter{tocdepth}{5}

%%%%%%%%%%%%%%%%%%%%%%%%%%%%%%%%%%%%%%%%%%%
%%%%%%%%%%%%%%%%%%%%%%%%%%%%%%%%%%%%%%%%%%%
%%%%%%%%%%%%FONTS%%%%%%%%%%%%%%%%%%%%%%%%%%
%%%%%%%%%%%%%%%%%%%%%%%%%%%%%%%%%%%%%%%%%%%
%%%%%%%%%%%%%%%%%%%%%%%%%%%%%%%%%%%%%%%%%%%

\defaultfontfeatures{Ligatures=TeX}
\setmainlanguage[babelshorthands=true]{russian}
\setotherlanguage{english}

\setmainfont{Constantia}

\setsansfont{Futura PT}[
	Path=C:/Windows/Fonts/,
	Extension		=	.otf,
	UprightFont		=	*Medium,
	ItalicFont		=	*MediumOblique,
	FontFace		=	{light}{n}{*Light},
	FontFace		=	{light}{it}{*LightOblique},
	FontFace		=	{light}{sl}{*LightOblique},		
	FontFace		=	{semibold}{n}{*Demi},
	FontFace		=	{semibold}{it}{*DemiOblique},
	BoldFont		=	*Heavy,
	BoldItalicFont	=	*HeavyOblique,
	FontFace		=	{extrabold}{n}{*Bold_0},
	FontFace		=	{extrabold}{it}{*BoldOblique},
	FontFace		=	{black}{n}{*ExtraBold},
	FontFace		=	{black}{it}{*ExtraBoldOblique},
]

\newfontfamily\cyrillicfont{Constantia}

\newfontfamily\cyrillicfontsf{Futura PT}[
	Path=C:/Windows/Fonts/,
	Extension		=	.otf,
	UprightFont		=	*Medium,
	ItalicFont		=	*MediumOblique,
	FontFace		=	{light}{n}{*Light},
	FontFace		=	{light}{it}{*LightOblique},
	FontFace		=	{light}{sl}{*LightOblique},		
	FontFace		=	{semibold}{n}{*Demi},
	FontFace		=	{semibold}{it}{*DemiOblique},
	BoldFont		=	*Heavy,
	BoldItalicFont	=	*HeavyOblique,
	FontFace		=	{extrabold}{n}{*Bold_0},
	FontFace		=	{extrabold}{it}{*BoldOblique},
	FontFace		=	{black}{n}{*ExtraBold},
	FontFace		=	{black}{it}{*ExtraBoldOblique},
]

\newfontfamily\cyrillicfonttt{Consolas}

\setmathfont[%
	version=scholamath,%
	math-style=ISO%
]{TeX Gyre Schola Math}


\makeatletter

\DeclareRobustCommand\blackseries
{\not@math@alphabet\blackseries\mathbf
	\fontseries{black}\selectfont
}

\DeclareRobustCommand\extraboldseries
{\not@math@alphabet\extraboldseries\mathbf
	\fontseries{extrabold}\selectfont
}

\DeclareRobustCommand\semiboldseries
{\not@math@alphabet\semiboldseries\mathbf
	\fontseries{semibold}\selectfont
}

\DeclareRobustCommand\mediumseries
{\not@math@alphabet\mediumseries\mathbf
	\fontseries{medium}\selectfont
}

\DeclareRobustCommand\lightseries
{\not@math@alphabet\lightseries\mathbf
	\fontseries{light}\selectfont
}

\DeclareRobustCommand\extralightseries
{\not@math@alphabet\extralightseries\mathbf
	\fontseries{extralight}\selectfont
}

\DeclareRobustCommand\thinseries
{\not@math@alphabet\thinseries\mathbf
	\fontseries{thin}\selectfont
}
\makeatother 

\DeclareTextFontCommand{\textblack}{\blackseries}
\DeclareTextFontCommand{\textexbf}{\extraboldseries} 
\DeclareTextFontCommand{\textsmbf}{\semiboldseries} 
\DeclareTextFontCommand{\textmed}{\mediumseries} 
\DeclareTextFontCommand{\textlg}{\lightseries} 
\DeclareTextFontCommand{\textexlg}{\extralightseries} 
\DeclareTextFontCommand{\textth}{\thinseries}

%%%%%%%%%%%%%%%%%%%%%%%%%%%%%%%%%%%%%%%%%%%
%%%%%%%%%%%%%%%%%%%%%%%%%%%%%%%%%%%%%%%%%%%
%%%%%%%%%%%%MICROTYPOGRAPHY%%%%%%%%%%%%%%%%
%%%%%%%%%%%%%%%%%%%%%%%%%%%%%%%%%%%%%%%%%%%
%%%%%%%%%%%%%%%%%%%%%%%%%%%%%%%%%%%%%%%%%%%


\SetProtrusion[%
	name		=	*%
]{%
	encoding	= 	{*},
	family		=	{*}
}{%
	\textendash			=	{0,250},% "---
	\textemdash			=	{0,250},
	-					=	{0,500},
	\textquoteleft		=	{1000,1000},% `
	\textquotedblleft	=	{1000,1000},% ``
	\glqq				=	{1000,0},%,, Opening 99 (down)
	\grqq				=	{0,1000},%`` closing 66 (up)
	{,}					=	{0,1000},
	{.}					=	{0,1000},
	{:}					=	{0,1000},
	{;}					=	{0,1000},
	{?}					=	{0,100},
	{)}					=	{0,100},
	{(}					= 	{150,0},
	\guillemotleft 		=	{200,0},%<<
	\guillemotright 	=	{0,200},%>>
}

%%%%%%%%%%%%%%%%%%%%%%%%%%%%%%%%%%%%%%%%%%%
%%%%%%%%%%%%%%%%%%%%%%%%%%%%%%%%%%%%%%%%%%%
%%%%%%%%%%%%LAYOUT%%%%%%%%%%%%%%%%%%%%%%%%%
%%%%%%%%%%%%%%%%%%%%%%%%%%%%%%%%%%%%%%%%%%%
%%%%%%%%%%%%%%%%%%%%%%%%%%%%%%%%%%%%%%%%%%%

\geometry{%
	%showframe,
	paperwidth		=	8.27in,
	paperheight		=	11.7in,
	bottom			=	2.6in,
	top				=	1.3in,
	inner			=	1.3in, 
	outer			=	2.6in,
	voffset			=	0in,
	marginparwidth	=	2in,
	marginparsep	=	0.3in,
	footskip		=	1in,
	headsep			=	0.3in,
	headheight		=	0.5in,
}

\renewcommand{\chaptertitlename}{Лекция}

\titleformat{\chapter}[display]
{\huge\sffamily\blackseries\raggedright}{\chaptertitlename\ \thechapter}{2ex}{\Huge}

\titleformat{\section}
{\Large\sffamily\blackseries\raggedright}{\thesection}{1em}{}

\titleformat{\section}
{\Large\sffamily\blackseries\raggedright}{\thesection}{1em}{}

\titleformat{\subsection}
{\large\sffamily\extraboldseries\raggedright}{\thesubsection}{1em}{}

\titleformat{\subsubsection}
{\normalsize\sffamily\bfseries\raggedright}{\thesubsubsection}{1em}{}

\titleformat{\paragraph}[hang]
{\normalsize\sffamily\bfseries\bfseries\raggedright}{\theparagraph}{1em}{}

\titleformat{\subparagraph}[hang]
{\normalsize\sffamily\mdseries\raggedright}{\thesubparagraph}{1em}{}

\titlespacing*{\chapter} {0pt}{50pt}{40pt}
\titlespacing*{\section} {0pt}{3.5ex plus 1ex minus .2ex}{2.3ex plus .2ex}
\titlespacing*{\subsection} {0pt}{3.25ex plus 1ex minus .2ex}{1.5ex plus .2ex}
\titlespacing*{\subsubsection}{0pt}{3.25ex plus 1ex minus .2ex}{1.5ex plus .2ex}
\titlespacing*{\paragraph}{0pt}{3.25ex plus 1ex minus .2ex}{0.5em}
\titlespacing*{\subparagraph}{0pt}{3.25ex plus 1ex minus .2ex}{0.25em}

\fancyhf{}

\fancyhead[RO]{\nouppercase{\scshape\liningnums{\leftmark}}}

\fancyhead[LO]{\nouppercase{\scshape\liningnums{\rightmark}}}

\fancyhead[RE]{}

\fancyhead[LE]{}

\fancyhead[CE]{}

\fancyhead[CO]{}

\fancyfoot[C]{\textup{\thepage}}

\fancyfoot[RO,LE]{}

\renewcommand\footrule{}

\renewcommand\headrule{%
	\begin{minipage}{1\textwidth}
		\hrule width \hsize height 2pt 
		\hrule width \hsize \kern 1mm
	\end{minipage}
}%

\fancypagestyle{plain}{%
	\fancyhf{}%
	\fancyhead[RO,RE,LE,LO,CO,CE]{}
	\fancyfoot[RO,LE,RE,LO]{}
	
	\fancyfoot[C]{\textup{\liningnums{\scshape\thepage}}}%\ из \pageref{LastPage}}}}
	
	
	\renewcommand\footrule{}%
	
	\renewcommand\headrule{}%
	
}

\pagestyle{fancy}

\renewcommand{\thefootnote}{\textfrak{\alphalph{\value{footnote}}}}

%%%%%%%%%%%%%%%%%%%%%%%%%%%%%%%%%%%%%%%%%%%
%%%%%%%%%%%%%%%%%%%%%%%%%%%%%%%%%%%%%%%%%%%
%%%%%%%%%%%%THEOREMS%%%%%%%%%%%%%%%%%%%%%%%
%%%%%%%%%%%%%%%%%%%%%%%%%%%%%%%%%%%%%%%%%%%
%%%%%%%%%%%%%%%%%%%%%%%%%%%%%%%%%%%%%%%%%%%


\renewcommand\qedsymbol{\bfseries\scshape QED}

\declaretheoremstyle[
	spaceabove=6pt,
	spacebelow=6pt,
	headindent=0pt,
	headfont=\normalfont\bfseries\scshape\upshape,
	notefont=\mdseries\scshape,
	notebraces={\unskip\kern.4em(}{)},
	headpunct=.,
	bodyfont=\normalfont\itshape,
	postheadspace=.4em,
]{THEOREMstyle}

\declaretheoremstyle[
	spaceabove=6pt,
	spacebelow=6pt,
	headindent=0pt,
	headfont=\normalfont\bfseries\scshape,
	notefont=\mdseries\scshape,
	notebraces={\unskip\kern.4em(}{)},
	headpunct=.,
	bodyfont=\normalfont\mdseries,
	postheadspace=.4em,
]{DEFINITIONstyle}


\declaretheoremstyle[
	spaceabove=4pt,
	spacebelow=10pt,
	headindent=0pt,
	headfont=\normalfont\bfseries\itshape,
	notefont=\mdseries\scshape,
	notebraces={\unskip\kern.4em(}{)},
	headpunct=.,
	bodyfont=\normalfont\mdseries,
	postheadspace=.4em,
	qed=\qedsymbol,
]{PROOFstyle}

\declaretheorem[%
	style=THEOREMstyle,%
	parent=chapter,%
	name=Теорема%
]{Theorem}

\declaretheorem[%
	style=THEOREMstyle,%
	sibling=Theorem,%
	name=Лемма%
]{Lemma}

\declaretheorem[%
	style=THEOREMstyle,%
	name=Аксиома%
]{Axiom}

\declaretheorem[%
	style=THEOREMstyle,%
	sibling=Theorem,%
	name=Утверждение%
]{Statement}

\declaretheorem[%
	style=THEOREMstyle,%
	sibling=Theorem,%
	name=Предложение%
]{Proposition}

\declaretheorem[%
	style=THEOREMstyle,%
	parent=Theorem,%
	name=Следствие%
]{Corollary}

\declaretheorem[%
	style=DEFINITIONstyle,%
	parent=chapter,%
	name=Определение%
]{Definition}

\declaretheorem[%
	style=DEFINITIONstyle,%
	numbered=no,%
	name=Замечание%
]{Remark}

\declaretheorem[%
	style=DEFINITIONstyle,%
	parent=chapter,%
	name=Упражнение%
]{Exercise}

\declaretheorem[%
	style=DEFINITIONstyle,%
	parent=chapter,%
	name=Задача%
]{Problem}

\declaretheorem[%
	style=DEFINITIONstyle,%
	parent=chapter,%
	name=Пример%
]{Example}

\declaretheorem[%
	style=PROOFstyle,%
	numbered=no,%
	name=Доказательство%
]{Proof}

\newcommand{\BTH}{\begin{Theorem}}
	\newcommand{\ETH}{\end{Theorem}}

\newcommand{\BAX}{\begin{Axiom}}
	\newcommand{\EAX}{\end{Axiom}}

\newcommand{\BLM}{\begin{Lemma}}
	\newcommand{\ELM}{\end{Lemma}}

\newcommand{\BPR}{\begin{Proposition}}
	\newcommand{\EPR}{\end{Proposition}}

\newcommand{\BST}{\begin{Statement}}
	\newcommand{\EST}{\end{Statement}}

\newcommand{\BCR}{\begin{Corollary}}
	\newcommand{\ECR}{\end{Corollary}}

\newcommand{\BDF}{\begin{Definition}}
	\newcommand{\EDF}{\end{Definition}}

\newcommand{\BEX}{\begin{Example}}
	\newcommand{\EEX}{\end{Example}}

\newcommand{\BPROB}{\begin{Problem}}
	\newcommand{\EPROB}{\end{Problem}}

\newcommand{\BRM}{\begin{Remark}}
	\newcommand{\ERM}{\end{Remark}}

\newcommand{\BPRF}{\begin{Proof}}
	\newcommand{\EPRF}{\end{Proof}}








